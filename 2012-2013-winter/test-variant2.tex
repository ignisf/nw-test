\documentclass[11pt]{scrartcl}

\usepackage[a4paper, margin=2.5cm]{geometry}

\usepackage{xecyr}
\usepackage{polyglossia}
\setmainlanguage{bulgarian}

\usepackage{libertineotf}

\usepackage{amsmath}
\newtheorem{pr}{Задача}
\newtheorem{q}{Въпрос}
\newcommand\ans{Отг.: }
\newcommand\rans{\hfill\ans}

\renewcommand{\theenumi}{\alph{enumi}}

\newenvironment{itemize*}%
  {\begin{itemize}%
    \setlength{\itemsep}{0pt}%
    \setlength{\parskip}{0pt}}%
  {\end{itemize}}

\newenvironment{defractors}{
\begin{enumerate}
  \setlength{\itemsep}{1pt}
  \setlength{\parskip}{0pt}
  \setlength{\parsep}{0pt}
}{\end{enumerate}}

\usepackage{hyperref}
\usepackage{booktabs}

\begin{document}

\title{}
\subtitle{\textrm{Вариант 2}}
\author{}
\date{}

\begin{center}
  Вариант №2
\end{center}

\begin{center}
Име, административна група, факултетен номер:

\end{center}

\begin{center}
..........................................................................................................................
\end{center}

\begin{q}
  За кой протокол от слой №4 на OSI модела не са характерни потвържденията
  (ack-nowledgements) и контрола на потока (flow control)?

  \begin{defractors}
  \item TCP
  \item UDP
  \item ICMP
  \item ARP
  \end{defractors}
\end{q}

\begin{q}
  Кое от изброените твърдения е вярно за каналния слой на OSI модела?

  \begin{defractors}
    \item Пакетите са продукт на каналния слой.
    \item Каналният слой е отговорен за сегментирането и реасемблирането на данните.
    \item Маршрутизирането е имплементирано в каналния слой.
    \item В каналния слой се дефинират хардуерните адреси на мрежовите интерфейси на хостовете.
  \end{defractors}
\end{q}

\begin{q}
  В кой слой на DoD (TCP/IP) модела е имплементирано маршрутизирането,
  позволяващо свързването и избирането на път за пренос на данни между две
  крайни системи?

  \begin{defractors}
  \item Приложен
  \item Сесиен
  \item Интернет
  \item Канален (Network Access)
  \end{defractors}
\end{q}

\begin{q}
  Кои слоеве на OSI модела биват обхванати от приложния слой на DoD (TCP/IP) модела?

  \begin{defractors}
    \item Транспортен, презентационен и сесиен.
    \item Приложен, сесиен и транспортен.
    \item Приложен, презентационен и сесиен.
    \item Мрежови, канален и интернет.
  \end{defractors}
\end{q}

\begin{q}
  Имате Linux-базирана система с един мрежови интерфейс, на който е зададен IP
  адресът \texttt{10.16.0.154/12}. До какво ще доведе изпълнението на командата
  \texttt{route add -net 10.0.0.0 netmask 255.240.0.0 gw 10.16.0.1}?

  \begin{defractors}
  \item Изпълнението на командата добавя в маршрутната таблица маршрут към
    подмрежата \texttt{10.0.0.0/12} през маршрутизатор с IP адрес
    \texttt{10.16.0.1}.
    \item Изпълнението на командата добавя маршрут по подразбиране през
      маршрутизатор с IP адрес \texttt{10.15.255.254}.
    \item Изпълнението на командата ще доведе до грешка.
    \item Изпълнението на командата води до добавяне на статичен маршрут до
      хоста с IP адрес \texttt{10.15.255.254}.
 \end{defractors}
\end{q}

\begin{q}
  Кои команди се използват в Linux за модифициране на маршрутната таблица?

  \begin{defractors}
    \item \texttt{arp}
    \item \texttt{iptables}
    \item \texttt{ip}
    \item \texttt{route}
  \end{defractors}
\end{q}

\begin{q}
  Кои флагове са вдигнати във втората протоколна единица от тристранното
  ръкостискане на TCP?

  \begin{defractors}
    \item \texttt{SYN}
    \item \texttt{RST}
    \item \texttt{ACK}
    \item Нито един от изброените.
  \end{defractors}
\end{q}


\begin{q}
  Какъв е максималният брой IP адреси, които могат да бъдат зачислени на хостове
  в локална подмрежа с маска \texttt{255.255.255.128}?
\end{q}

\begin{q}
  Колко подмрежи и колко адреса за хостове в подмрежа предоставя мрежовият
  адрес \texttt{192.168.1.0/25}? Посочете broadcast адреса за дадената подмрежа.
\end{q}

\begin{q}
  Колко адреса за хостове предлага подмрежа с 27-битова маска?
\end{q}

\begin{q}
  Напишете адреса на подмрежата, broadcast адреса на подмрежата и интер- вала от
  валидни адреси на хостове за следния адрес \texttt{192.168.100.25} с маска
  \texttt{255.255.255.252}.
\end{q}

\begin{q}
  Имате клас B мрежа и се нуждаете от 25 подмрежи. Каква мрежова маска ще
  изберете? Колко валидни хостови адреса бихте получили в отделните подмрежи?
\end{q}

\begin{q}
  Посочете верните твърдения относно NAT.

  \begin{defractors}
  \item Спестява публично достъпни IP адреси.
  \item Отразява се негативно върху сигурността на мрежата.
  \item Причинява усложнения в комуникацията на някои приложения.
  \item Нарушава принципа на end-to-end свързаност в Интернет.
  \end{defractors}
\end{q}

\begin{q}
  Посочете неверните твърдения относно NAT.
  \begin{defractors}
  \item Отразява се негативно върху сигурността на мрежата.
  \item Ускорява предаването на данни в Интернет.
  \item Нарушава принципа на end-to-end свързаност в Интернет.
  \item Спестява публично достъпни IP адреси.
  \end{defractors}
\end{q}

\begin{q}
  Идентифицирайте слоя от DoD (TCP/IP) модела, към който принадлежи всеки един
  от следните протоколи:

  \begin{defractors}
    \item Internet Protocol (IP)
    \item Telnet User Datagram Protocol (UDP)
    \item Ethernet File
    \item Transfer Protocol (FTP)
  \end{defractors}
\end{q}

\begin{q}
  Посочете характерните протоколни единици за пренос на данни (Protocol Data
  Units) за следните протоколи:

  \begin{defractors}
  \item Internet Protocol (IP)
  \item Ethernet
  \item Transmission Control Protocol (TCP)
  \end{defractors}
\end{q}
\end{document}